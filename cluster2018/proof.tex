\subsection{Optimal set of precisions}

In this subsection, we provide an optimal set of precisions to minimize the
total execution time, under the assumption that the time evolves linearly with
$b$ the number of precision bits: $T(b) = \alpha b + c$. \leo{This assumption
is supported by current technologies, in which double precision executions take
twice the time to complete than single precision executions. This is due to
processors having more low precision FPUs than high precision FPUs (See
previous section).}

   \begin{theorem}

       Given $b_{max}$, a maximum number of bits, $n$ the number of different
       precisions $b_1 \leq b_2 \leq \dots \leq b_{n-1} \leq b_n$ to use, and $T(b)=\alpha b+c$, with $\alpha$
       and $c$ two constants, the time to execute a cycle at precision $b$,
       then the execution time of our adaptive algorithm is minimized for $\forall 1\leq i\leq n, b_i
       = \frac{i}{n}b_{max}$.

   \end{theorem}

\begin{proof}

    From the previous experiments we can see that (i) the number of cycles
    needed to reach the lower bound for a given precision does not depend on
    the precision used during previous cycles (See Figure~\ref{fig.prec_incr}),
    i.e.  if one starts with cycles at precision $16$ bits and then switch to
    $32$ bits, you will need the same number of cycles to reach the lower bound
    with $32$ bits as if you used only cycles with $32$ bits from the beginning
    of the algorithm; and that (ii) the number of cycles needed to reach the
    lower bound is proportional to the number of bits $b$ used (See
    Figure~\ref{fig.bits_accuracy}). We then define $\textsc{MaxIter}(b)$ the
    number of cycles needed so that the ratio between the relative
    residual norms computed before and after these cycles is higher than a threshold $t$. 
    We use the two observations
    to model $\textsc{MaxIter}(b) = \lfloor kb \rfloor$ for some constant $k$.
    Then we can compute the total execution time:

    \begin{align*}
        T_{total}& = \textsc{MaxIter}(b_1)T(b_1)\\ & +  \sum\limits_{i=1}^{n-1}
        (\textsc{MaxIter}(b_{i+1})-\textsc{MaxIter}(b_i))T(b_{i+1}).
    \end{align*}

    Indeed, when we reach the number of iterations $\textsc{MaxIter}(b_i)$ we
    change from precision $b_i$ to precision $b_{i+1}$. This means that until we are before the iteration $\textsc{MaxIter}(b_1)$, we compute using $b_1$ bits. Then until we reach the iteration $\textsc{MaxIter}(b_2)$, we compute using $b_2$ bits, that is during $\textsc{MaxIter}(b_2) - \textsc{MaxIter}(b_1)$ iterations, and so on until we reach the maximum accuracy with $b_n$ bits at iteration $\textsc{MaxIter}(b_n)$.  We can rewrite
    $T_{total}$ as

    \begin{align*}
        T_{total} &\approx k b_{1} T(b_1) + \sum\limits_{i=1}^{n-1}
        k(b_{i+1}-b_{i})T(b_{i+1})\\ & \approx k ( b_{n}T(n) +
        \sum\limits_{i=1}^{n-1} b_i ( T(b_i) - T(b_{i+1})).
    \end{align*}

    By plugging the expression of $T(b)$ into the previous equation and considering
    the maximum precision we want is $b_{max}=b_n$, we finally get:

    \begin{equation}
        T_{total}  = k\alpha\left(b_n^2 + \sum\limits_{i=1}^{n-1} (b_i^2 - b_i b_{i+1})\right) + kb_{max}c.
    \end{equation}

    Let us consider the function $f(x_1,\dots,x_n) = \sum\limits_{i=1}^n x_i^2
    - \sum\limits_{i=1}^{n-1} x_ix_{i+1}$. Finding the minimum of
    $f(x_1,\dots,x_{n-1},1)$ will give us the minimum of the execution time.

    By simple partial derivation: \[ \frac{\partial f(x_1,\dots,x_n)}{\partial
    x_i} = 2x_i - (1-\delta_{i,n})x_{i+1} - (1-\delta_{i,1})x_{i-1} \] where
    $\delta_{a,b} $ is equal to 1 if and only if $a=b$, 0 otherwise.

    This is where the boundary condition $x_n = 1$ is useful: if we do not set it
    to a number different from 0, the function is minimized for
    $x_1=\dots=x_n=0$. Applying this to our problem makes no sense, as we
    would not be doing any computation. The boundary condition represents the
    fact that we need to reach an existing precision ($b_{max}$) eventually. By
    a simple scaling, considering $1$ instead of $b_{max}$ makes computation
    easier and does not change the resolution of the the following system (up
    to a factor):

    \resizebox{0.9\linewidth}{!}{
    $\left\{
    \begin{tabular}{rcl cccccccccc cl}
    $\frac{\partial f(x_1,\dots,x_n)}{\partial x_1}$ & = & $2x_1$ & $-$ & $x_2$  &     &       & & 	 &&     &     &       & = & 0 \\
    $\frac{\partial f(x_1,\dots,x_n)}{\partial x_2}$ & = & $-x_1$ & $+$ & $2x_2$ & $-$ & $x_3$ & & 	 &&     &     &       & = & 0 \\
    & \vdots &&&&&&&&&&&&&\\
    $\frac{\partial f(x_1,\dots,x_{n})}{\partial x_{n-1}}$ & = &        &     &        &     &       & & $-x_{n-2}$ & $+$ & $2x_{n-1}$ & $-$ & $x_n$ & = & 0 \\
    $x_n$ & = & 1 &&&&&&&&&&&&\\
    \end{tabular} \right.$
    }

    Solving this system of equations is equivalent to solving the linear system $Ax=b$ where
    \[ A =
    \begin{bmatrix}
    2       & -1 &  &  &  \\
    -1       & 2 & -1 &  &  \\
    & \ddots & \ddots & \ddots & \\
    & & -1 & 2 & -1 \\
           &  &  & -1 & 2
    \end{bmatrix}
    \textup{ and } b = \begin{bmatrix} 0 \\ \vdots \\ 0 \\ 1 \end{bmatrix} \]

    This system has a unique solution which is $\begin{bmatrix} \frac{1}{n} & \dots
    & \frac{n-1}{n} \end{bmatrix}$.  The minimum of $f(x_1,\dots,x_n)$ with
    boundary condition $x_n=1$ is thus reached for $x_i = \frac{i}{n}$.  It
    is, when multiplied by $b_{max}$, the solution that minimizes our total
    execution time.

\end{proof}

   This proof holds only if the execution time evolves linearly with $b$, but
   \textit{we could also apply it to minimize the energy consumption} assuming
   that the energy consumption of one cycle increases linearly with $b$. In the
   next section we investigate whether this assumption holds.

