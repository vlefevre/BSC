\section{Multi-Grid Algorithm}
\label{sec:algo}

\leo{A very brief introduction to the MG algorithm.. }

%\subsection{Definitions}

%\begin{itemize}
% \item A system of equations is represented by the following equation: $Ax=b$, where $A \in \mathcal{M}(\mathbb{R})^{n\times n}$ and $b \in \mathcal{\mathbb{R}}^n$ are given and
% $x \in \mathbb{R}^n$ is the unknown. The \emph{exact} solution of this system will be denoted by $\widetilde{x}$.
% \item A level is an integer between $1$ and $L$. Level $1$ will be called the finest level, and level $L$ will be called the coarsest level.
%  \item The restriction of $A$ (or $b$ or $x$) to level $l$ will be denoted by $A^l$ (or $b^l$ or $x^l$). We have $A^1 = A$ (and $b^1=b,x^1=x$).
%  \item We define a set of $L-1$ restriction matrices $R_1,\dots,R_{L-1}$ such that $R_l b^l = b^{l+1}$. We also define some prolongation matrices $P_1,\dots,P_{L-1}$ such that $P_{l}b^{l+1} = b^l$.
%  In other words, we have $P_l = {R_l}^{-1}$ (left inverse) and we build the $A^l$ matrices as follows: $A^{l+1} = R_l A^l P_l$.
%  \item We denote by $e^l$ the error at level $l$, that is the vector such that $x^l + e^l = \widetilde{x^l}$, that is to say $\widetilde{x^l}-x^l$.
%  We also define the residual at level $l$, $r^l = b^l - A^lx^l$. As $b^l = A^l\widetilde{x^l}$, we can also write $r^l = A^le^l$.
%  \item We derive the relative residual norm at any step $i$ in the algorithm
%  by the norm of the residual at this step, $||b^l - A^lx^l_i||$, divided by the norm of the initial residual, $|| b^l - A^lx^l_0||$.\\ We also define the notion of \emph{tolerance}
%  as an real value between 0 and 1, which is a threshold for stopping an algorithm. In multi-grid algorithms, this threshold will be on the residual norm.
%  \item We call relaxation a step of an iterative method for solving linear systems (such as Jacobi, Gauss-Sneidel, \dots). Formally, for a vector $x \in \mathbb{R}^n$, it represents the computation of
%  $x \leftarrow Mx + c$ where $M \in \mathcal{M}(\mathbb{R})^{n\times n}$ and $c \in \mathcal{\mathbb{R}}^n$ are defined depending on the method used.
% \end{itemize}

%\subsection{The V-cycle}
%
%  The goal of the algorithm is to improve the efficiency of iterative methods. Indeed, the choice of the starting vector $x$ on which to apply relaxations has consequences on the convergence
%  time of the solver, and depending on the system to solve, the factor of convergence (related to the matrix $M$) can be close to 1.\\
%  Here the idea is to do some relaxations and then correct the value of $x$ by adding to it the corresponding error term. However, this error term cannot be computed easily (otherwise,
%  solving the problem would be done by computing the error term and adding it to $x$). Multi-grid solvers instead use recursion to compute the error term. The stopping parameter for the
%  recursion will be determined by decreasing the sizes of vectors and matrices (thus loosing some accuracy but saving time).
%  Formally, we can sum up the algorithm as follows:
%
%  MG$(l,x,f,\alpha_1,\alpha_2)$:
%  \begin{itemize}
%    \item If $l = L$, return $x = {A^L}^{-1} f$ (exact solve);
%    \item Else:
%    \begin{enumerate}
%      \item Relax $x$ $\alpha_1$ times using an iterative method (matrix $A^l$, right hand side $f$);
%      \item $r \leftarrow R_l ( f - Ax )$;
%      \item $y \leftarrow 0$:
%      \item MG$(l+1,y,r,\alpha_1,\alpha_2)$;
%      \item $e \leftarrow P_{l} y$;
%      \item $x \leftarrow x+e$;
%      \item Relax $x$ $\alpha_2$ times using an iterative method (matrix $A^l$, right hand side $f$);
%   \end{enumerate}
%  \end{itemize}
%  The algorithm is then executed by setting $x^l \leftarrow 0$ and then executing MG$(1,x^l,b^l,\alpha_1,\alpha_2)$.
%
%  Then several ways of modifying the algorithm appear:
%  \begin{itemize}
%   \item Which iterative method to use?
%   \item Do we want only one recursive call at each level or more?
%   \item How many times do we need to apply the algorithm?
%   \item How to determine good $\alpha_1$ and $\alpha_2$ parameters?
%   \item How many levels should be defined?
%  \end{itemize}
%
%  In all what follows the iterative method chosen is an hybrid Jacobi/Gauss-Seidel method. The number of levels used will not be studied.



