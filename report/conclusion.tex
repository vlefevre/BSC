\section{Conclusion}
\label{sec:conclusions}

This paper improves the original MG algorithm by two different ways: The first
one is to remove some relaxation steps in a V-cycle, which leads to faster
cycles although requires more of them to converge.  Overall, this solution
achieves up to 30\% improvement although its performance enhancements vary a
lot depending on the workload, The second way to improve the algorithm is to
adapt the precisions of the floating point depending on MG's precision
requirements, We use low precisions levels during the very first MG's cycles
and we increase them as the execution progresses.  Repeating this operation
until reaching the maximum available accuracy available leads to significant
speedups.  We estimate the benefits of these two approaches combined on
different scenarios.  When combining half-, single- and double-precisions, we
can reduce by 16.4\% and 14.5\% the execution time compared to using double- or
single-precision, respectively, during the whole execution.

In the future we plan to explore additional ideas to reduce the execution time
even more like changing the precision used in different levels of a cycle.

%However, we think that it is not useful because (1) using a greater precision
%in the coarse levels than in the fine levels is useless as all the
%computations would eventually be truncated and (2) using a smaller precision
%in the coarse levels than in the fine levels would not affect by much the
%execution time as, according to the measurements done, only the first 2 or 3
%levels represent more than 95\% of the relaxation cost of a cycle.

Also, we plan to estimate the impact of our techniques on the energy
consumption by obtaining some measurements on real machines.  Evaluating the
energy consumption using different values of the $\alpha$ parameter is also in
our immediate future plans.


