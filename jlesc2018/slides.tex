\documentclass{beamer}
\addtobeamertemplate{footline}{\insertframenumber/\inserttotalframenumber}

\usepackage[utf8]{inputenc}
\usepackage[american]{babel}
\usepackage{default}
\usepackage{amsfonts,amsmath,amssymb,amsthm,amstext,latexsym}
\usepackage{listings}
\usepackage{tikz}
\usepackage{lmodern}
\usepackage{bold-extra}
\usepackage{multirow}
\usepackage{verbatim}
\usetikzlibrary{calc}
\usetikzlibrary{arrows,shapes}
\usetikzlibrary{patterns,snakes}

\usepackage{hyperref}
\usepackage[absolute,overlay]{textpos}
\usepackage{xspace}
%\usepackage{subcaption}
\usepackage{array,multirow,makecell}
\usepackage{todonotes}
\usepackage{ifthen}
\usepackage{intcalc}
%\usepackage{fullpage}
\usepackage[ruled,vlined,boxed,linesnumbered,commentsnumbered]{algorithm2e}
\usepackage{subfig}
\usepackage{paralist}

\definecolor{mygray}{gray}{0.9}
%\definecolor{red}{100,0,0}
\newcommand<>{\red}[1]{{\color#2{red!80!black}#1}}
\usetheme{Madrid}

\title{Approximating a Multi-Grid solver}


\author[Valentin Le Fèvre]{\red{Valentin Le Fèvre} \and Leonardo Bautista-Gomez \and Marc Casas\\
{\small \texttt{valentin.le-fevre@ens-lyon.fr}}}

\date{April 17, 2018}

\begin{document}
 
\begin{frame}

\maketitle
\end{frame}

\section{Introduction}

\begin{frame}{Introduction}
  present the multi-grid algorithm, approximate computing definitions...
\end{frame}

\section{Analysis}

\begin{frame}{Execution times}
  show the measurements
\end{frame}

\begin{frame}{maybe another one}
\end{frame}

\section{The \textsc{Up}-cycle}

\begin{frame}{The idea}
  explain what we do
\end{frame}

\begin{frame}{Results}
  show numbers/figures that show the improvement
\end{frame}

\section{Changing the bitwidth}

\begin{frame}{Impact on convergence rate}
  show that reducing the bitwidth does not change accuracy until we reach the threshold

\end{frame}

\begin{frame}{Algorithm}
  explain our adaptive algorithm

\end{frame}

\begin{frame}{Execution time model}
  model formula and methodology
\end{frame}

\begin{frame}{Results}
  bar plot which show the improvement

\end{frame}

\section{Conclusion}

\begin{frame}{Conclusion}

  final result with combination of both techniques

  open on energy consumption estimation techniques... link to silent data corruption

\end{frame}
 
 
\end{document}
