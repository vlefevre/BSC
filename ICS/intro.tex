\section{Introduction}
\label{sec:intro}

Multi-Grid (MG) solvers are a class of linear methods~\cite{Hackbusch1991} that
emerged in the 80's to increase the convergence rate of more classical
iterative methods.  They became even more important with the advent of very
complex scientific applications~\cite{Ashby1996}, which required very powerful
linear solvers.  The usage of MG solvers has become a common practice in
today's parallel systems due to the good scalability properties this methods
display, which have been detailedly analyzed and modeled~\cite{Gahvari11}.
Also, MG solvers have been reported to display more robustness against silent
data corruptions than traditional iterative methods deployed over a single
grid~\cite{Casas12}, which also implies they behave well under reduced accuracy
scenarios.

MG solvers rely on a grid of evaluation points that discretize the
domain of a continuous differential equation.  Typically, MG schemes are
defined by coarsening a fine-grain grid until reaching a small set of
evaluation points where a direct method can be applied.
%This grid is represented using different granularity levels that are used to
%determine a solution on the inaccurate and coarse grids.
Solutions obtained on the inaccurate and coarse grids are used on the more
accurate and fine-grain levels to accelerate the process of obtaining the final
solution of the system.
%For example, if we want to solve an equation of the form: \[ -u^{''}(x)+au(x)
%= f(x),\quad 0<x<1, \quad u(0)=u(1)=0, \] where $a$ is some constant, $f$ a
%function and $u$ an unknown function, we can transform it into a discrete
%problem by selecting $N$ evaluation points $x_1,\dots,x_N$ and using taylor
%expansion to express the second derivative term $u^{''}(x_i)$ as a weighted
%sum of $u(x_j)$ for some values of $j$.  Then to define coarser grids, a
%simple way to do it is to divide the number of evaluation points and transform
%the original equation into another discrete problem (thus using evaluation
%points $x_1,x_3,\dots,x_N$).  Typically, multi-grid schemes are defined by
%coarsening a fine-grain grid until reaching a small enough set of evaluation
%points on which a direct solve would be fast.
Multi-grid solvers typically consist of three phases: The \textit{Relaxation},
\textit{Restriction} and \textit{Interpolation} phases.  The relaxation phase
applies a few steps of an iterative solver like Jacobi or Gauss-Seidel at a
certain coarseness level.  The restriction phase propagates the algorithmic
state to a coarser grid by means of linear transformations while the
interpolation phase maps the coarser estimate to the finer version and adjusts
the current solution with the new error.

One common way of orchestrating a Multi-Grid Solver execution is via a V-cycle,
where we first iterate on the finest grid, then the second finest grid and so
on until reaching the coarsest grid where a direct solve is used instead of an
iterative method as the problem size has become smaller.  Then we iterate again
on all the other grids in the reverse order to have a solution expressed with
the initial granularity of the grid.  Different parameters, such as the iterative
method used at each step or the granularity of the grid and the previously
mentioned cycle shape, affect the convergence rate of the algorithm.

%Typically, multi-grid schemes are defined by coarsening a fine-grain grid
%until reaching a small enough set of evaluation points on which a direct solve
%would be fast.  Multi-grid solvers then perform several steps of an iterative
%method like Jacobi~\cite{} or Gauss-Seidel~\cite{} switching between the
%different grids by interpolating or restricting the solution of the previous
%step.  More precisely, a multi-grid algorithm has a parameter which is the
%cycle shape.  The cycle shape determines in which order and on which
%coarsening of the grid an iterative method is used. The classical cycle shape
%is the V-cycle where we first iterate on the finest grid, then the second
%finest grid and so on until reaching the coarsest grid where a direct solve is
%used instead of an iterative method as the problem size has become smaller.
%Then we iterate again on all the other grids in the reverse order to have a
%solution expressed with the initial fineness of the grid.  Different
%parameters, such as the iterative method used at each step or the fineness of
%the grid and the previously mentioned cycle shape, affect the convergence rate
%of the algorithm. For example, using a grid with only a few evaluation points
%decreases the execution time but deteriorates the accuracy of the final
%result. In the other way, it is possible to increase the quality of the result
%by using a finer grid but at the cost of a performance degradation.

In this context, this paper investigates different trade-offs between accuracy
and execution time for MG algorithms. This paper focuses its effort on
one of the most popular parallel implementation of a multi-grid solver, the
BoomerAMG~\cite{boomerAMG}, implemented in the HYPRE
library~\cite{Falgout2002}.  This paper makes the following contributions
beyond the state-of-the-art:

\begin{itemize}

    \item We evaluate the performance impact of parameters of the solver such as the shape
        of cycles and the number of iterations.

    \item We propose a new cycle configuration and show that it is more
        efficient than the state-of-the-art configurations by up to 28.3\%.

    \item We evaluate the impact of different floating-point precisions on
        the time to completion while reaching full accuracy on the final
        result.

    \item We propose an algorithm that dynamically adapts the precision of the
        MG algorithm variables to increase performance and efficiency.

    \item We perform a large evaluation and demonstrate the we can reduce by
        15\% the execution time needed to reach the same quality result as the
        original double-precision algorithm and up to 30\% for smaller
        accuracies.

\end{itemize}


The reminder of this paper is organized as follows:
Section~\ref{sec:motivation} introduces the motivations for this work.
Section~\ref{sec:pruning} investigates the impact of different parameters on
the time to completion.  Section~\ref{sec:precision} explores
precision-performance trade-offs.  Section~\ref{sec:related} discuss other
related works and Section~\ref{sec:conclusions} concludes this paper.

