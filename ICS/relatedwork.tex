\section{Related Work}
\label{sec:related}

Approximate computing proposals exploit in many different ways the capacity of
some computer programs to deliver correct results while operating at low
accuracy scenarios: Precision scaling consists in reducing the precision of
inputs or intermediate values to reduce storage or computing
operands~\cite{Yeh2007, Tian2015}.  Other techniques consist in skipping some
loop iterations to reduce computational load~\cite{Sidiroglou2011}.  
Such approaches are reported to provide performance improvements over a factor of two
while impacting the final application's accuracy by less than 10\%.
Load Value
Approximation (LVA) consists in exploiting the redundant nature of some
computer codes to predict load values and let the execution to progess without
stalling~\cite{Miguel2014}.  Memoization, which is an approach consisting in
storing function results to predict the result of subsequent computations, it
is being used to achieve performance enhancements and energy consumption
reductions~\cite{Alvarez2005, Brumar2017}. 
Fast but slightly incorrect
hardware adders have also been explored 
for video and image compression algorithms~\cite{Gupta:2011}. 
Post-layout simulations show improvements of up to 60\% in terms of power and area savings of up to 37\%
without significant output quality loss.
Voltage-scaling techniques have also been explored to reduce energy consumption at the SRAM level, which increases
the probability of bit flips~\cite{Sampson:2011}. 
Changing the bit-width of
some variables has been proposed~\cite{Park:2010} to trade image quality for
bandwidth.

Some techniques have been proposed to accelerate MG algorithms.  A strategy
consisting in   carrying out some relaxation steps while going down the V-cycle
and interpolating them to finer grid levels has been proposed~\cite{JAMESON}.
It is aimed at damping perturbations in the linear system solution, even if
direct interpolation introduces some degree of error.


